%-----------------------------------------------------
% index key words
%-----------------------------------------------------
\index{plane}

%-----------------------------------------------------
% name, leave blank
% title, if the exercise has a name i.e. Hilbert's matrix
% difficulty = n, where n is the number of stars
% origin = "\cite{ref}"
%-----------------------------------------------------
\begin{Exercise}[
name={},
title={}, 
difficulty=0,
origin={\cite{JH}}]
\Question Describe the plane through \( (1,1,5,-1) \),
           \( (2,2,2,0) \), and \( (3,1,0,4) \).
\Question Is the origin in that plane?

\end{Exercise}

\begin{Answer}
\Question Note that
          \begin{equation*}
            \colvec[r]{2 \\ 2 \\ 2 \\ 0}
            -\colvec[r]{1 \\ 1 \\ 5 \\ -1}
            =\colvec[r]{1 \\ 1 \\ -3 \\ 1}
            \qquad
            \colvec[r]{3 \\ 1 \\ 0 \\ 4}
            -\colvec[r]{1 \\ 1 \\ 5 \\ -1}
            =\colvec[r]{2 \\ 0 \\ -5 \\ 5}
          \end{equation*}
          and so the plane is this set.
          \begin{equation*}
            \set{\colvec[r]{1 \\ 1 \\ 5 \\ -1}
                 +\colvec[r]{1 \\ 1 \\ -3 \\ 1}t
                 +\colvec[r]{2 \\ 0 \\ -5 \\ 5}s
                \suchthat t,s\in\Re}
          \end{equation*}
\Question No; this system
          \begin{equation*}
            \begin{linsys}{3}
              1  &+  &1t  &+  &2s  &=  &0  \\
              1  &+  &1t  &   &    &=  &0  \\
              5  &-  &3t  &-  &5s  &=  &0  \\
             -1  &+  &1t  &+  &5s  &=  &0  
            \end{linsys}
          \end{equation*}
          has no solution.

\end{Answer}

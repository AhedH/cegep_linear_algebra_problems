%-----------------------------------------------------
% index key words
%-----------------------------------------------------
\index{norm}
\index{unit vector}

%-----------------------------------------------------
% name, leave blank
% title, if the exercise has a name i.e. Hilbert's matrix
% difficulty = n, where n is the number of stars
% origin = "\cite{ref}"
%-----------------------------------------------------
\begin{Exercise}[
name={},
title={}, 
difficulty=0,
origin={\cite{JH}}]
    Show that if \( \vec{v}\neq\zero \) then
    \( \vec{v}/\norm{\vec{v}\,} \) has length one.
    What if \( \vec{v}=\zero \)?
\end{Exercise}

\begin{Answer}
     Assume that \( \vec{v}\in\Re^n \) has components \( v_1,\ldots,v_n \).
      If \( \vec{v}\neq \zero \) then we have this.
      \begin{multline*}
        \sqrt{\left(\frac{v_1}{\sqrt{{v_1}^2+\cdots+{v_n}^2}}\right)^2+
              \dots+\left(\frac{v_n}{\sqrt{{v_1}^2+\cdots+{v_n}^2}}\right)^2}
        \\
        \begin{aligned}
          &=\sqrt{\left(\frac{{v_1}^2}{{v_1}^2+\cdots+{v_n}^2}\right)+
              \dots+\left(\frac{{v_n}^2}{{v_1}^2+\cdots+{v_n}^2}\right)}   \\
          &=1
        \end{aligned}
      \end{multline*}
      If \( \vec{v}=\zero \) then \( \vec{v}/\norm{\vec{v}\,} \) is not
      defined.
\end{Answer}

%-----------------------------------------------------
% index key words
%-----------------------------------------------------
\index{intersection:of two planes}

%-----------------------------------------------------
% name, leave blank
% title, if the exercise has a name i.e. Hilbert's matrix
% difficulty = n, where n is the number of stars
% origin = "\cite{ref}"
%-----------------------------------------------------
\begin{Exercise}[
name={},
title={}, 
difficulty=0,
origin={\cite{JH}}]
    Find the intersection of these planes.
    \begin{align*}
      \set{\colvec[r]{1 \\ 1 \\ 1}t+
           \colvec[r]{0 \\ 1 \\ 3}s
           \suchthat t,s\in\Re}\\
      \set{\colvec[r]{1 \\ 1 \\ 0}
           +\colvec[r]{0 \\ 3 \\ 0}k+
           \colvec[r]{2 \\ 0 \\ 4}m
           \suchthat k,m\in\Re}
    \end{align*}
\end{Exercise}

\begin{Answer}
     The points of coincidence are solutions of this system.
      \begin{equation*}
        \begin{linsys}{2}
         t  &  &   &= &1+2m\hfill      \\
         t  &+ &s  &= &1+3k\hfill      \\
         t  &+ &3s &= &4m\hfill
        \end{linsys}
      \end{equation*}
      Using Gaussian elimination on
      \begin{equation*}
        \begin{amat}{4}
          1  &0  &0  &-2  &1  \\
          1  &1  &-3 &0   &1  \\
          1  &3  &0  &-4  &0
        \end{amat}
	\end{equation*}
 we get
	\begin{equation*}
        \begin{amat}{4}
          1  &0  &0  &-2  &1  \\
          0  &1  &-3 &2   &0  \\
          0  &0  &9  &-8  &-1
        \end{amat}
      \end{equation*}
      which gives \( k=-(1/9)+(8/9)m \), so \( s=-(1/3)+(2/3)m \) and \( t=1+2m \).
      The intersection is
      \begin{equation*}
        \set{\colvec[r]{1 \\ 1 \\ 0}+
             \colvec[r]{0 \\ 3 \\ 0}(-\frac{1}{9}+\frac{8}{9}m)+
             \colvec[r]{2 \\ 0 \\ 4}m
             \suchthat m\in\Re}
        =\set{\colvec[r]{1 \\ 2/3 \\ 0}
             +\colvec[r]{2 \\ 8/3 \\ 4}m
             \suchthat m\in\Re}
      \end{equation*}    
\end{Answer}

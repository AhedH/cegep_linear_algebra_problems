%-----------------------------------------------------
% index key words
%-----------------------------------------------------
\index{orthogonal vector}

%-----------------------------------------------------
% name, leave blank
% title, if the exercise has a name i.e. Hilbert's matrix
% difficulty = n, where n is the number of stars
% origin = "\cite{ref}"
%-----------------------------------------------------
\begin{Exercise}[
name={},
title={}, 
difficulty=0,
origin={\cite{JH}}]
    Show that \( \norm{\vec{u}\,}=\norm{\vec{v}\,} \)
    if and only if \( \vec{u}+\vec{v} \) and
    \( \vec{u}-\vec{v} \) are perpendicular.
    Give an example in \( \Re^2 \).
\end{Exercise}

\begin{Answer}
      Where \( \vec{u},\vec{v}\in\Re^n \), the vectors
      \( \vec{u}+\vec{v} \) and \( \vec{u}-\vec{v} \) are perpendicular if and
      only if
      $0=(\vec{u}+\vec{v})\dotprod(\vec{u}-\vec{v})
         =\vec{u}\dotprod\vec{u}-\vec{v}\dotprod\vec{v}$,
      which shows that those two are perpendicular if and only if
      \( \vec{u}\dotprod\vec{u}=\vec{v}\dotprod\vec{v} \).
      That holds if and only if \( \norm{\vec{u}\,}=\norm{\vec{v}\,} \).
\end{Answer}

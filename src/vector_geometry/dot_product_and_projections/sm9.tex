%-----------------------------------------------------
% index key words
%-----------------------------------------------------
\index{norm}
\index{angle}
\index{unit vector}

%-----------------------------------------------------
% name, leave blank
% title, if the exercise has a name i.e. Hilbert's matrix
% difficulty = n, where n is the number of stars
% origin = "\cite{ref}"
%-----------------------------------------------------
\begin{Exercise}[
name={},
title={}, 
difficulty=0,
origin={\cite{SM}}]
Let $\vec{v}$ be any non-zero vector in $\Re^2$, and $\hat{v}$ be the unit vector in its direction.
\Question Show that $\hat{v}$ can be written as $\hat{v}=(\cos\phi,\sin\phi)$, where $\phi$ is the angle from the positive $x$-axis to $\vec{v}$. Also show that $\vec{v}=\norm{\vec{v}}(\cos\phi,\sin\phi)$.
\Question Prove the formula $\cos(\alpha-\beta)=\cos\alpha\cos\beta+\sin\alpha\sin\beta$ by considering the dot product of the two unit vectors $\vec{e_a}=(\cos\alpha,\sin\alpha)$ and $\vec{e_b}=(\cos\beta,\sin\beta)$.

\end{Exercise}

\begin{Answer}
\Question If $\vec{v} = (v_1,v_2)$, show that $v_1 = \norm{\vec{v}}\cos\phi$ and that $v_2 = \norm{\vec{v}}\sin\phi$. This shows that $\vec{v}=\norm{\vec{v}}(\cos\phi,\sin\phi)$, and since $\hat{v} = \frac{\vec{v}}{\norm{\vec{v}}}$, the required formula follows.
\Question The angle between the vectors $\vec{e_a}$ and $\vec{e_b}$ is exactly $\alpha-\beta$, so by the definition of the dot product we get $\cos(\alpha-\beta) = \dfrac{\vec{e_a}\dotprod\vec{e_b}}{\norm{\vec{e_a}}\norm{\vec{e_b}}}$. The denominator here is one since they are both unit vectors, and working out the top gives the required result.
\end{Answer}

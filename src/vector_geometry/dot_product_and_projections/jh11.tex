%-----------------------------------------------------
% index key words
%-----------------------------------------------------
\index{Pythagorean theorem}

%-----------------------------------------------------
% name, leave blank
% title, if the exercise has a name i.e. Hilbert's matrix
% difficulty = n, where n is the number of stars
% origin = "\cite{ref}"
%-----------------------------------------------------
\begin{Exercise}[
name={},
title={}, 
difficulty=0,
origin={\cite{JH}}]
   Generalize to \( \Re^n \) the converse of the Pythagorean
    Theorem, that
    if \( \vec{u} \) and \( \vec{v} \) are
    perpendicular then
    \( \norm{\vec{u}+\vec{v}\,}^2=\norm{\vec{u}\,}^2+\norm{\vec{v}\,}^2 \).
\end{Exercise}

\begin{Answer}
      Suppose that \( \vec{u},\vec{v}\in\Re^n \).
      If \( \vec{u} \) and \( \vec{v} \) are perpendicular then
      \begin{equation*}
        \norm{\vec{u}+\vec{v}\,}^2
        =(\vec{u}+\vec{v})\dotprod(\vec{u}+\vec{v})    
        =\vec{u}\dotprod\vec{u}+2\,\vec{u}\dotprod\vec{v}
               +\vec{v}\dotprod\vec{v}  
        =\vec{u}\dotprod\vec{u}+\vec{v}\dotprod\vec{v}  
        =\norm{\vec{u}\,}^2+\norm{\vec{v}\,}^2
      \end{equation*}  
      (the third equality holds because \( \vec{u}\dotprod\vec{v}=0 \)).
\end{Answer}

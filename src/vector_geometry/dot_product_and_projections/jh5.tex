%-----------------------------------------------------
% index key words
%-----------------------------------------------------
\index{norm}

%-----------------------------------------------------
% name, leave blank
% title, if the exercise has a name i.e. Hilbert's matrix
% difficulty = n, where n is the number of stars
% origin = "\cite{ref}"
%-----------------------------------------------------
\begin{Exercise}[
name={},
title={}, 
difficulty=0,
origin={\cite{JH}}]
    Does any vector have
    length zero except a zero vector?
    (If ``yes'', produce an example.
    If ``no'', prove it.)
\end{Exercise}

\begin{Answer}
      We prove that a vector has length zero if and only if all its
      components are zero.

      Let \( \vec{u}\in\Re^n \) have components \( u_1,\ldots,u_n \).
      Recall that the square of any real number is greater than or equal to
      zero, with equality only when that real is zero.
      Thus
      \( \absval{\vec{u}\,}^2={u_1}^2+\cdots+{u_n}^2 \) is a sum of numbers
      greater than or equal to zero, and so is itself greater than or equal
      to zero, with equality if and only if each \( u_i \) is zero.
      Hence \( \absval{\vec{u}\,}=0 \) if and only if all the components of
      \( \vec{u} \) are zero.    
\end{Answer}

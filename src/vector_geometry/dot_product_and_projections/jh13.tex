%-----------------------------------------------------
% index key words
%-----------------------------------------------------
\index{bisector}

%-----------------------------------------------------
% name, leave blank
% title, if the exercise has a name i.e. Hilbert's matrix
% difficulty = n, where n is the number of stars
% origin = "\cite{ref}"
%-----------------------------------------------------
\begin{Exercise}[
name={},
title={}, 
difficulty=0,
origin={\cite{JH}}]
    Prove that, where \( \vec{u},\vec{v}\in\Re^n \) are nonzero vectors,
    the vector
    \begin{equation*}
       \frac{\vec{u}}{\norm{\vec{u}\,} }+\frac{\vec{v}}{\norm{\vec{v}\,} }
    \end{equation*}
    bisects the angle between them.
    Illustrate in \( \Re^2 \).
\end{Exercise}

\begin{Answer}
      We will show something more general:~if
      \( \norm{\vec{z}_1}=\norm{\vec{z}_2} \) for
      \( \vec{z}_1,\vec{z}_2\in\Re^n \), then \( \vec{z}_1+\vec{z}_2 \)
      bisects the angle between \( \vec{z}_1 \) and \( \vec{z}_2 \)
      \begin{center}
        \setlength{\unitlength}{4pt}      % sum of equal length vectors.
        \begin{picture}(37,12)(0,0)
          \put(0,0){\begin{picture}(12,12)(0,0)
                      \thicklines
                      \put(0,0){\vector(2,1){8} }
                      \put(0,0){\vector(1,2){4} }
                      \put(0,0){\vector(1,1){12} }
                      \thinlines
                      \put(8,4){\line(1,2){4} }
                      \put(4,8){\line(2,1){8} }
                    \end{picture} }
          \put(18.5,7){\makebox(0,0){\small gives} }
          \put(25,0){\begin{picture}(12,12)(0,0)
                      \thinlines
                      \put(0,0){\line(2,1){8} }
                      \put(0,0){\line(1,2){4} }
                      \put(0,0){\line(1,1){12} }
                      \put(8,4){\line(1,2){4} }
                      \put(4,8){\line(2,1){8} }

                      \put(2,3.8){\( \prime \)}
                      \put(4.2,1.5){\( \prime \)}
                      \multiput(5.7,5.2)(0.5,0.5){2}{\( \prime \)}
                      \multiput(9.0,5.8)(0.5,1.0){3}{\( \prime \)}
                      \multiput(6.6,8.7)(0.6,0.3){3}{\( \prime \)}
                    \end{picture} }
        \end{picture}
      \end{center}
      (we ignore the case where \( \vec{z}_1 \) and \( \vec{z}_2 \) are
      the zero vector).

      The \( \vec{z}_1+\vec{z}_2=\zero \) case is easy.
      For the rest, by the definition of angle, 
      we will be finished if we show this.
      \begin{equation*}
        \frac{\vec{z}_1\dotprod(\vec{z}_1+\vec{z}_2)}{
              \norm{\vec{z}_1}\,\norm{\vec{z}_1+\vec{z}_2} }
        =
        \frac{\vec{z}_2\dotprod(\vec{z}_1+\vec{z}_2)}{
              \norm{\vec{z}_2}\,\norm{\vec{z}_1+\vec{z}_2} }
      \end{equation*}
      But distributing inside each expression gives
      \begin{equation*}
        \frac{\vec{z}_1\dotprod\vec{z}_1+\vec{z}_1\dotprod\vec{z}_2}{
              \norm{\vec{z}_1}\,\norm{\vec{z}_1+\vec{z}_2} }
        \qquad
        \frac{\vec{z}_2\dotprod\vec{z}_1+\vec{z}_2\dotprod\vec{z}_2}{
              \norm{\vec{z}_2}\,\norm{\vec{z}_1+\vec{z}_2} }
      \end{equation*}
      and \( \vec{z}_1\dotprod\vec{z}_1=\norm{\vec{z}_1}^2
              =\norm{\vec{z}_2}^2=\vec{z}_2\dotprod\vec{z}_2 \), so the
      two are equal.  

\end{Answer}

%-----------------------------------------------------
% index key words
%-----------------------------------------------------
\index{}


%-----------------------------------------------------
% name, leave blank
% title, if the exercise has a name i.e. Hilbert's matrix
% difficulty = n, where n is the number of stars
% origin = "by name \cite{ref}"
%-----------------------------------------------------
\begin{Exercise}[
name={},
title={}, 
difficulty=0,
origin={\cite{JH}}]
Prove that, where \( a,b,\ldots,e \) are real numbers
and \( a\neq 0 \), if
\begin{equation*}
ax+by=c
\end{equation*}
has the same solution set as
\begin{equation*}
ax+dy=e
\end{equation*}
then they are the same equation.
What if \( a=0 \)?
\end{Exercise}

\begin{Answer}
If \( a\neq 0 \) then the solution set of the first equation is
\( \set{(x,y)\suchthat x=(c-by)/a} \).
Taking $y=0$ gives the solution $(c/a,0)$, and since the second
equation is supposed to have the same solution set, substituting into
it gives that $a(c/a)+d\cdot 0=e$, so $c=e$.
Then taking $y=1$ in $x=(c-by)/a$ gives that $a((c-b)/a)+d\cdot 1=e$,
which gives that $b=d$.
Hence they are the same equation.

When \( a=0 \) the equations can be different and still have the 
same solution set:~e.g.,
\( 0x+3y=6 \) and \( 0x+6y=12 \).   
\end{Answer}

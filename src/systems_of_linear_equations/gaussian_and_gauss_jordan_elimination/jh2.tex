%-----------------------------------------------------
% index key words
%-----------------------------------------------------
\index{unique solution, inconsistent, infinitely many solutions}


%-----------------------------------------------------
% name, leave blank
% title, if the exercise has a name i.e. Hilbert's matrix
% difficulty = n, where n is the number of stars
% origin = "by name \cite{ref}"
%-----------------------------------------------------
\begin{Exercise}[
name={},
title={}, 
difficulty=0,
origin={\cite{JH}}]
For which values of \( k \) are
there no solutions, many solutions, or a unique solution
to this system?
\begin{equation*}
\begin{linsys}{2}
x  &-  &y  &=  &1  \\
3x  &-  &3y &=  &k  
\end{linsys}
\end{equation*}

\end{Exercise}

\begin{Answer}
After performing Gaussian elimination the system becomes
\begin{equation*}
\begin{linsys}{2}
x  &-  &y  &=  &1\hfill  \\
   &   &0  &=  &-3+k\hfill  
\end{linsys}
\end{equation*}
This system has no solutions if \( k\neq 3 \) and if
\( k=3 \) then it has infinitely many solutions.
It never has a unique solution.  
\end{Answer}

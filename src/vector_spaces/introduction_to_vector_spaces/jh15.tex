%-----------------------------------------------------
% index key words
%-----------------------------------------------------
\index{vector space}
\index{zero vector}

%-----------------------------------------------------
% name, leave blank
% title, if the exercise has a name i.e. Hilbert's matrix
% difficulty = n, where n is the number of stars
% origin = "\cite{ref}"
%-----------------------------------------------------
\begin{Exercise}[
name={},
title={}, 
difficulty=0,
origin={\cite{JH}}]
The definition of vector spaces does not explicitly say that
\( \zero+\vec{v}=\vec{v} \)
(it instead says that \( \vec{v}+\zero=\vec{v} \)).
Show that it must nonetheless hold in any vector space.

\end{Exercise}

\begin{Answer}

Addition is commutative, so in any vector space,
for any vector \( \vec{v} \) we have that
\( \vec{v}=\vec{v}+\zero=\zero+\vec{v} \).

\end{Answer}

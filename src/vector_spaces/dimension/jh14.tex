%-----------------------------------------------------
% index key words
%-----------------------------------------------------
\index{dimension}

%-----------------------------------------------------
% name, leave blank
% title, if the exercise has a name i.e. Hilbert's matrix
% difficulty = n, where n is the number of stars
% origin = "\cite{ref}"
%-----------------------------------------------------
\begin{Exercise}[
name={},
title={}, 
difficulty=0,
origin={\cite{JH}}]
      \Question Consider first how bases might be related by $\subseteq$.
        Assume that \( U,W \) are subspaces of some vector space and
        that \( U\subseteq W \).
	
	Can there exist bases \( B_U \) for \( U \) and \( B_W \) for \( W \)
        such that \( B_U\subseteq B_W \)?
        Must such bases exist?

        For any basis \( B_U \) for \( U \), must there be a basis \( B_W \)
        for \( W \) such that \( B_U\subseteq B_W \)?

        For any basis \( B_W \) for \( W \), must there be a basis \( B_U \)
        for \( U \) such that \( B_U\subseteq B_W \)?

        For any bases \( B_U, B_W \) for \( U \) and \( W \), must  \( B_U \)
        be a subset of \( B_W \)?
      \Question Is the $\cap$ of bases a basis?
        For what space?
      \Question Is the $\cup$ of bases a basis?
        For what space?
      \Question What about the complement operation?
\end{Exercise}

\begin{Answer}
       First, note that a set is a basis for some space if and only
       if it is linearly independent, because in that case 
       it is a basis for its own span.
         \Question The answer to the question in the second paragraph
            is ``yes''
           (implying ``yes'' answers for both questions in the first
           paragraph).
           If \( B_U \) is a basis for \( U \) then
           \( B_U \) is a linearly independent subset of \( W \).
           It is possible to expand it to a basis for 
           \( W \).
           That is the desired \( B_W \).

           The answer to the question in the third paragraph is ``no'', which
           implies a ``no'' answer to the question of the fourth paragraph.
           Here is an example of a basis for a superspace with no sub-basis
           forming a basis for a subspace: in \( W=\Re^2 \), consider the
           standard basis \( \stdbasis_2 \).
           No sub-basis of $\stdbasis_2$ forms a basis for the 
           subspace \( U \) 
           of $\Re^2$ that is the line \( y=x \).
         \Question It is a basis (for its span) because the
           intersection of linearly
           independent sets is linearly independent (the intersection is a
           subset of each of the linearly independent sets).

           It is not, however, a basis for the intersection of the spaces.
           For instance, these are bases for \( \Re^2 \):
           \begin{equation*}
             B_1=\sequence{\colvec[r]{1 \\ 0},\colvec[r]{0 \\ 1}}
             \quad\text{and}\quad
             B_2=\sequence[r]{\colvec{2 \\ 0},\colvec[r]{0 \\ 2}}
           \end{equation*}
           and \( \Re^2\intersection\Re^2=\Re^2 \), but
           \( B_1\intersection B_2 \) is empty.
           All we can say is that the $\cap$ of the bases is a basis
           for a subset of the intersection of the spaces.
         \Question The $\cup$ of bases need not be a basis: in \( \Re^2 \)
           \begin{equation*}
             B_1=\sequence{\colvec[r]{1 \\ 0},\colvec[r]{1 \\ 1}}
             \quad\text{and}\quad
             B_2=\sequence{\colvec[r]{1 \\ 0},\colvec[r]{0 \\ 2}}
           \end{equation*}
           \( B_1\union B_2 \) is not linearly independent.
           A necessary and sufficient condition for a $\cup$ of two bases
           to be a basis 
           \begin{equation*}
             B_1\union B_2 \text{ is linearly independent }
             \quad\iff\quad
             \spanof{B_1\intersection B_2}=\spanof{B_1}\intersection
                                            \spanof{B_2}
           \end{equation*}
           it is easy enough to prove (but perhaps hard to apply).
         \Question The complement of a basis cannot be a basis
           because it contains the zero vector.

\end{Answer}

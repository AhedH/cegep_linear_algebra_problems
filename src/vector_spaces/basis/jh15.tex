%-----------------------------------------------------
% index key words
%-----------------------------------------------------
\index{basis}
\index{symmetric matrix}
\index{matrix!symmetric}

%-----------------------------------------------------
% name, leave blank
% title, if the exercise has a name i.e. Hilbert's matrix
% difficulty = n, where n is the number of stars
% origin = "\cite{ref}"
%-----------------------------------------------------
\begin{Exercise}[
name={},
title={}, 
difficulty=0,
origin={\cite{JH}}]
\Question Find a basis for the vector space of
        symmetric \( \nbyn{2} \) matrices.
\Question Find a basis for the space of symmetric \( \nbyn{3} \)
        matrices.
\Question Find a basis for the space of symmetric \( \nbyn{n} \)
        matrices.
\end{Exercise}

\begin{Answer}
       \Question Describing the vector space as
          \begin{equation*}
             \set{\begin{mat}
                     a  &b  \\
                     b  &c
                  \end{mat}  \suchthat a,b,c\in\Re}
          \end{equation*}
          suggests this for a basis.
          \begin{equation*}
            \sequence{
              \begin{mat}[r]
                1  &0  \\
                0  &0
              \end{mat},
              \begin{mat}[r]
                0  &0  \\
                0  &1
              \end{mat},
              \begin{mat}[r]
                0  &1  \\
                1  &0
              \end{mat}  }
          \end{equation*}
          Verification is easy.
        \Question This is one possible basis.
          \begin{equation*}
            \sequence{
              \begin{mat}[r]
                1  &0  &0  \\
                0  &0  &0  \\
                0  &0  &0
              \end{mat},
              \begin{mat}[r]
                0  &0  &0  \\
                0  &1  &0  \\
                0  &0  &0
              \end{mat},
              \begin{mat}[r]
                0  &0  &0  \\
                0  &0  &0  \\
                0  &0  &1
              \end{mat},
              \begin{mat}[r]
                0  &1  &0  \\
                1  &0  &0  \\
                0  &0  &0
              \end{mat},
              \begin{mat}[r]
                0  &0  &1  \\
                0  &0  &0  \\
                1  &0  &0
              \end{mat},
              \begin{mat}[r]
                0  &0  &0  \\
                0  &0  &1  \\
                0  &1  &0
              \end{mat}  }
          \end{equation*}
         \Question As in the prior two questions, we can form a basis from two
           kinds of  matrices.
           First are the matrices with a single one on the diagonal and all
           other entries zero (there are \( n \) of those matrices).
           Second are the matrices with two opposed off-diagonal entries
           are ones and all other entries are zeros.
           (That is, all entries in $M$ are zero except that 
           $m_{i,j}$ and $m_{j,i}$ are one.) 

\end{Answer}

%-----------------------------------------------------
% index key words
%-----------------------------------------------------
\index{basis}
\index{polynomial space}

%-----------------------------------------------------
% name, leave blank
% title, if the exercise has a name i.e. Hilbert's matrix
% difficulty = n, where n is the number of stars
% origin = "\cite{ref}"
%-----------------------------------------------------
\begin{Exercise}[
name={},
title={}, 
difficulty=0,
origin={\cite{JH}}]
Find the span of each set \textit{(that is, find restriction(s) on the coefficient of the polynomial)} and then find a basis for that span.
\Question $\set{1+x,\;1+2x}$ in $\polyspace_2$
\Question $\set{2-2x,\;3+4x^2}$ in $\polyspace_2$
\end{Exercise}

\begin{Answer}
\Question Asking which $a_0+a_1x+a_2x^2$ can be expressed as
           $c_1\cdot (1+x)+c_2\cdot (1+2x)$ 
           gives rise to three linear equations,
           describing the coefficients of $x^2$, $x$, and the constants.
           \begin{equation*} 
             \begin{linsys}{2}
               c_1 &+ &c_2  &= &a_0 \\
               c_1 &+ &2c_2 &= &a_1 \\
                   &  &0    &= &a_2
             \end{linsys}
           \end{equation*}
           Gauss's Method with back-substitution shows, 
           provided that $a_2=0$, that $c_2=-a_0+a_1$
           and $c_1=2a_0-a_1$.
           Thus, with $a_2=0$, we can compute appropriate
           $c_1$ and $c_2$ for any $a_0$ and $a_1$.
           So the span is the entire set of linear polynomials
           $\set{a_0+a_1x\suchthat a_0,a_1\in\Re}$.
           Parametrizing that set 
           $\set{a_0\cdot 1+a_1\cdot x\suchthat a_0,a_1\in\Re}$
           suggests a basis $\sequence{1,x}$ 
           (we've shown that it spans; checking linear independence is easy). 
\Question With
          \begin{equation*}
            a_0+a_1x+a_2x^2
            =c_1\cdot(2-2x)+c_2\cdot(3+4x^2)
            =(2c_1+3c_2)+(-2c_1)x+(4c_2)x^2
          \end{equation*}
          we get this system.
           \begin{equation*} 
             \begin{linsys}{2}
               2c_1  &+ &3c_2  &= &a_0 \\
               -2c_1 &  &      &= &a_1 \\
                     &  &4c_2  &= &a_2
              \end{linsys}
	\end{equation*}
	and Gaussian elimination gives
        \begin{equation*}     
	\begin{linsys}{2}
               2c_1  &+ &3c_2  &= &a_0\hfill\hbox{} \\
                     &  &3c_2  &= &a_0+a_1\hfill\hbox{} \\
                     &  &0     &= &(-4/3)a_0-(4/3)a_1+a_2
              \end{linsys}
           \end{equation*}
           Thus, the only quadratic polynomials $a_0+a_1x+a_2x^2$ with 
           associated $c$'s are the ones such that 
           $0=(-4/3)a_0-(4/3)a_1+a_2$.
           Hence the span is this.
           \begin{equation*}
             \set{(-a_1+(3/4)a_2)+a_1x+a_2x^2\suchthat a_1,a_2\in\Re}
           \end{equation*}
           Parametrizing gives
           $\set{a_1\cdot (-1+x)+a_2\cdot ((3/4)+x^2)\suchthat a_1,a_2\in\Re}$,
           which suggests $\sequence{-1+x,(3/4)+x^2}$
           (checking that it is linearly independent is routine).

\end{Answer}

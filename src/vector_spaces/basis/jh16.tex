%-----------------------------------------------------
% index key words
%-----------------------------------------------------
\index{basis}

%-----------------------------------------------------
% name, leave blank
% title, if the exercise has a name i.e. Hilbert's matrix
% difficulty = n, where n is the number of stars
% origin = "\cite{ref}"
%-----------------------------------------------------
\begin{Exercise}[
name={},
title={}, 
difficulty=0,
origin={\cite{JH}}]
We can show that every basis for $\Re^3$ contains the same
number of vectors.
\Question Show that no linearly independent subset of $\Re^3$
          contains more than three vectors.
\Question Show that 
          no spanning subset of $\Re^3$ contains fewer than three vectors.
          \textit{Hint:} 
          recall how to calculate the span of a set and show that
          this method 
          cannot yield all of $\Re^3$ when we apply it to fewer than
          three vectors.
\end{Exercise}

\begin{Answer}
\Question Any four vectors from $\Re^3$ are linearly related because
           the vector equation
           \begin{equation*}
             c_1\colvec{x_1 \\ y_1 \\ z_1}
             +c_2\colvec{x_2 \\ y_2 \\ z_2}
             +c_3\colvec{x_3 \\ y_3 \\ z_3}
             +c_4\colvec{x_4 \\ y_4 \\ z_4}
             =\colvec[r]{0 \\ 0 \\ 0}
           \end{equation*}
           gives rise to a linear system
           \begin{equation*}
             \begin{linsys}{4}
                x_1c_1  &+  &x_2c_2  &+  &x_3c_3  &+  &x_4c_4  &=  &0 \\
                y_1c_1  &+  &y_2c_2  &+  &y_3c_3  &+  &y_4c_4  &=  &0 \\
                z_1c_1  &+  &z_2c_2  &+  &z_3c_3  &+  &z_4c_4  &=  &0
             \end{linsys}
           \end{equation*}
           that is homogeneous (and so has a solution) and has 
           four unknowns but only three equations,
           and therefore has nontrivial solutions.
           (Of course, this argument applies to any subset of $\Re^3$ 
           with four or more vectors.) 
\Question We shall do just the two-vector case.
           Given $x_1$, \ldots, $z_2$, 
           \begin{equation*}
             S=\set{\colvec{x_1 \\ y_1 \\ z_1},
             \colvec{x_2 \\ y_2 \\ z_2}} 
           \end{equation*}
           to decide which vectors
           \begin{equation*}
             \colvec{x \\ y \\ z}
           \end{equation*}
           are in the span of $S$, set up 
           \begin{equation*}
             c_1\colvec{x_1 \\ y_1 \\ z_1}
             +c_2\colvec{x_2 \\ y_2 \\ z_2}
             =\colvec{x \\ y \\ z}
           \end{equation*}
           and row reduce the resulting system.
           \begin{equation*}
             \begin{linsys}{2}
                x_1c_1  &+  &x_2c_2   &=  &x \\
                y_1c_1  &+  &y_2c_2   &=  &y \\
                z_1c_1  &+  &z_2c_2   &=  &z
             \end{linsys}
           \end{equation*}
           There are two
           variables $c_1$ and $c_2$ but three equations, so 
           when Gauss's Method finishes, on the bottom
           row there will be some relationship of the form
           $0=m_1x+m_2y+m_3z$. 
           Hence, vectors in the span of the two-element set $S$
           must satisfy some restriction.
           Hence the span is not all of $\Re^3$.
\end{Answer}

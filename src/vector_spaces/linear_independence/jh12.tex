%-----------------------------------------------------
% index key words
%-----------------------------------------------------
\index{linear independence}
\index{linear dependence}

%-----------------------------------------------------
% name, leave blank
% title, if the exercise has a name i.e. Hilbert's matrix
% difficulty = n, where n is the number of stars
% origin = "\cite{ref}"
%-----------------------------------------------------
\begin{Exercise}[
name={},
title={}, 
difficulty=0,
origin={\cite{JH}}]
\Question Prove that a set of two perpendicular 
        nonzero vectors from
        \( \Re^n \) is linearly independent when \( n>1 \).
\Question What if \( n=1 \)?
\Question Generalize to more than two vectors.
\end{Exercise}

\begin{Answer}
         \Question Assume that \( \vec{v} \) and \( \vec{w} \) are
           perpendicular nonzero vectors in $\Re^n$, with \( n>1 \).
           With the linear relationship \( c\vec{v}+d\vec{w}=\zero \), 
           apply \( \vec{v} \) to both
           sides to conclude that \( c\cdot\norm{\vec{v}}^2+d\cdot 0=0 \).
           Because \( \vec{v}\neq\zero \) we have that \( c=0 \).
           A similar application of \( \vec{w} \) shows that \( d=0 \).
         \Question Two vectors in \( \Re^1 \) are perpendicular if and only if
           at least one of them is zero.
         \Question The right generalization is to look at a set
           \( \set{\vec{v}_1,\dots,\vec{v}_n}\subseteq\Re^k \) of vectors
           that are \definend{mutually orthogonal} 
           (also called \definend{pairwise perpendicular}):~if
           \( i\neq j \) then \( \vec{v}_i \) is perpendicular to
           \( \vec{v}_j \).
           Mimicking the proof of the first item above shows that such a set of
           nonzero vectors is linearly independent.
\end{Answer}

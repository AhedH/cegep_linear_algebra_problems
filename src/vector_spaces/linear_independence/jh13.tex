%-----------------------------------------------------
% index key words
%-----------------------------------------------------
\index{linear independence}
\index{linear dependence}

%-----------------------------------------------------
% name, leave blank
% title, if the exercise has a name i.e. Hilbert's matrix
% difficulty = n, where n is the number of stars
% origin = "\cite{ref}"
%-----------------------------------------------------
\begin{Exercise}[
name={},
title={}, 
difficulty=0,
origin={\cite{JH}}]
Show that, where \( S \) is a subspace of \( V \), if a subset $T$ of
    \( S \) is linearly independent in \( S \) then $T$ is also linearly
    independent in \( V \).
    Is the converse also true?

\end{Exercise}

\begin{Answer}
      It is both `if' and `only if'.

      Let \( T \) be a subset of the subspace \( S \) of the vector space
      \( V \).
      The assertion that any linear relationship 
      $c_1\vec{t}_1+\dots+c_n\vec{t}_n=\zero$ among members of \( T \)
      must be the trivial relationship $c_1=0$, \ldots, $c_n=0$
      is a statement that 
      holds in \( S \) if and only if it holds in \( V \),
      because the subspace \( S \) inherits its addition and 
      scalar multiplication operations from \( V \).  
\end{Answer}

%-----------------------------------------------------
% index key words
%-----------------------------------------------------
\index{linear independence}
\index{linear dependence}
\index{polynomial space}

%-----------------------------------------------------
% name, leave blank
% title, if the exercise has a name i.e. Hilbert's matrix
% difficulty = n, where n is the number of stars
% origin = "\cite{ref}"
%-----------------------------------------------------
\begin{Exercise}[
name={},
title={}, 
difficulty=0,
origin={\cite{JH}}]
Which of these subsets of \( \polyspace_2 \) are
    linearly dependent and which are independent?
\Question \( \set{3-x+9x^2,\;5-6x+3x^2,\;1+1x-5x^2} \)
\Question \( \set{-x^2,\;1+4x^2} \)
\Question \( \set{2+x+7x^2,\;3-x+2x^2,\;4-3x^2} \)
\Question \( \set{8+3x+3x^2,\;x+2x^2,\;2+2x+2x^2,\;8-2x+5x^2} \)
\end{Exercise}

\begin{Answer}
\Question This set is independent.
          Setting up the relation
          \( c_1(3-x+9x^2)+c_2(5-6x+3x^2)+c_3(1+1x-5x^2)=0+0x+0x^2 \)
          gives a linear system 
          \begin{equation*}
            \begin{amat}[r]{3}
              3  &5  &1  &0  \\
              -1 &-6 &1  &0  \\
              9  &3  &-5 &0  
            \end{amat}
        \end{equation*}
and Gaussian elimination gives
\begin{equation*}    
	\begin{amat}[r]{3}
              3  &5   &1        &0  \\
              0  &-13 &4        &0  \\
              0  &0   &-128/13  &0  
            \end{amat}
          \end{equation*}
          with only one solution: \( c_1=0 \), \( c_2=0 \), and \( c_3=0 \).
        \Question This set is independent.
           We can see this by inspection, straight from the definition
           of linear independence.
           Obviously neither is a multiple of the other.
        \Question This set is linearly independent.
           The linear system reduces in this way
           \begin{equation*}
             \begin{amat}[r]{3}
               2  &3  &4  &0  \\
               1  &-1 &0  &0  \\
               7  &2  &-3 &0  
             \end{amat}
        \end{equation*}
and Gaussian elimination gives
\begin{equation*}                
\begin{amat}[r]{3}
               2  &3    &4      &0  \\
               0  &-5/2 &-2     &0  \\
               0  &0    &-51/5  &0  
             \end{amat}
           \end{equation*}
           to show that there is only the solution $c_1=0$, 
           $c_2=0$, and $c_3=0$.
        \Question This set is linearly dependent.
           The linear system
           \begin{equation*}
             \begin{amat}[r]{4}
               8  &0  &2  &8  &0  \\ 
               3  &1  &2  &-2 &0  \\
               3  &2  &2  &5  &0
             \end{amat}
           \end{equation*}
           must, after reduction, end with at least one variable free
           (there are more variables than equations, and there is no
           possibility of a contradictory equation because the system is
           homogeneous).
           We can take the free variables as parameters to describe the
           solution set.
           We can then set the parameter to a nonzero value to get a
           nontrivial linear relation. 

\end{Answer}

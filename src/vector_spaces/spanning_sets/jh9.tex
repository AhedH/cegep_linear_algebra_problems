%-----------------------------------------------------
% index key words
%-----------------------------------------------------
\index{subspace}
\index{spaning set}

%-----------------------------------------------------
% name, leave blank
% title, if the exercise has a name i.e. Hilbert's matrix
% difficulty = n, where n is the number of stars
% origin = "\cite{ref}"
%-----------------------------------------------------
\begin{Exercise}[
name={},
title={}, 
difficulty=0,
origin={\cite{JH}}]
    Because `span of' is an operation on sets we naturally consider
    how it interacts with the usual set operations.
      \Question If \( S\subseteq T \) are subsets of a vector space, is
        \( \spanof{S}\subseteq\spanof{T} \)?
      \Question If \( S,T \) are subsets of a vector space, is
        \( \spanof{S\union T}=\spanof{S}\union\spanof{T} \)?
      \Question If \( S,T \) are subsets of a vector space, is
        \( \spanof{S\intersection T}=\spanof{S}\intersection\spanof{T} \)?
      \Question Is the span of the complement equal to the complement of
        the span?

\end{Exercise}

\begin{Answer}
         \Question Always;
           if \( S\subseteq T \) then a linear combination of elements of
           \( S \) is also a linear combination of elements of \( T \).
         \Question Sometimes (more precisely, if and only if 
           \( S\subseteq T \) or \( T\subseteq S \)).

           The answer is not `always' as is shown by this example from
           \( \Re^3 \)
           \begin{equation*}
             S=\set{\colvec[r]{1 \\ 0 \\ 0},\colvec[r]{0 \\ 1 \\ 0}},\quad
             T=\set{\colvec[r]{1 \\ 0 \\ 0},\colvec[r]{0 \\ 0 \\ 1}}
           \end{equation*}
           because of this.
           \begin{equation*}
             \colvec[r]{1 \\ 1 \\ 1}\in\spanof{S\union T}
             \qquad
             \colvec[r]{1 \\ 1 \\ 1}\not\in\spanof{S}\union \spanof{T}
           \end{equation*}

           The answer is not `never' because if either set contains the other
           then equality is clear.
           We can
           characterize equality as happening only when either set contains
           the other by assuming \( S\not\subseteq T \) (implying the
           existence of a vector \( \vec{s}\in S \) with 
           \( \vec{s}\not\in T \))
           and \( T\not\subseteq S \) (giving a \( \vec{t}\in T \) with
           \( \vec{t}\not\in S \)), noting
           \( \vec{s}+\vec{t}\in\spanof{S\union T} \),
           and showing that 
           \( \vec{s}+\vec{t}\not\in\spanof{S}\union\spanof{T} \).
         \Question Sometimes.

           Clearly
           \( \spanof{S\intersection T}
             \subseteq\spanof{S}\intersection\spanof{T} \)
           because any linear combination of vectors from 
           \( S\intersection T \)
           is a combination of vectors from \( S \) and also a combination of
           vectors from \( T \).

           Containment the other way does not always hold.
           For instance, in \( \Re^2 \), take
           \begin{equation*}
             S=\set{\colvec[r]{1 \\ 0},\colvec[r]{0 \\ 1}},\quad
             T=\set{\colvec[r]{2 \\ 0}}
           \end{equation*}
           so that \( \spanof{S}\intersection\spanof{T} \) is the \( x \)-axis
           but \( \spanof{S\intersection T}  \) is the trivial subspace.

           Characterizing exactly when equality holds is tough.
           Clearly equality holds if either set contains the other, but that is
           not `only if' by this example in \( \Re^3 \).
           \begin{equation*}
             S=\set{\colvec[r]{1 \\ 0 \\ 0},\colvec[r]{0 \\ 1 \\ 0}},
             \quad
             T=\set{\colvec[r]{1 \\ 0 \\ 0},\colvec[r]{0 \\ 0 \\ 1}}
           \end{equation*}
        \Question Never, as the span of the complement is a subspace, while
          the complement of the span is not (it does not contain the zero 
          vector).


\end{Answer}

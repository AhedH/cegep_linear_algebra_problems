\section{Matrix Inverses and Elementary Matrices}
%-----------------------------------------------------
% index key words
%-----------------------------------------------------
\index{matrix!skew-symmetric}
\index{matrix!symmetric}

%-----------------------------------------------------
% name, leave blank
% title, if the exercise has a name i.e. Hilbert's matrix
% difficulty = n, where n is the number of stars
% origin = "\cite{ref}"
%-----------------------------------------------------
\begin{Exercise}[
name={},
title={}, 
difficulty=0,
origin={\cite{GH}}]
\Question Prove that for any $n \times n$ matrix $A$, $A + A^T$ is symmetric and $A-A^T$ is skew-symmetric.
\Question Prove that any $n \times n$ can be written as the sum of a symmetric and skew-symmetric matrices. 
\end{Exercise}

\begin{Answer}
\Question Verify using the definition of symmetric and skew-symmetric matrices.
\Question Hint: use the previous part as the symmetric and skew-symmetric matrices.
\end{Answer}

%-----------------------------------------------------
% index key words
%-----------------------------------------------------
\index{unique solution}
\index{Gaussian elimination}
\index{inconsistent}
\index{infinite solutions}

%-----------------------------------------------------
% name, leave blank
% title, if the exercise has a name i.e. Hilbert's matrix
% difficulty = n, where n is the number of stars
% origin = "by name \cite{ref}"
%-----------------------------------------------------
\begin{Exercise}[
name={},
title={}, 
difficulty=0,
origin={\cite{GH}}]
State for which values of $k$ the given system will have exactly 1 solution, infinite solutions, or no solution.
\begin{multicols}{2}
\Question 
$
\begin{linsys}{2}
x_1&+&2x_2&=&1\\2x_1&+&4x_2&=&k
\end{linsys}
$
\Question
$
\begin{linsys}{2}
x_1&+&2x_2&=&1\\x_1&+&kx_2&=&1
\end{linsys}
$
\Question
$
\begin{linsys}{2}
x_1&+&2x_2&=&1\\x_1&+&kx_2&=&2
\end{linsys}
$
\Question
$
\begin{linsys}{2}
x_1&+&2x_2&=&1\\x_1&+&3x_2&=&k
\end{linsys}
$
\EndCurrentQuestion
\end{multicols}
\end{Exercise}

\begin{Answer}
\Question Never exactly 1 solution; infinite solutions if $k=2$; no solution if $k\neq 2$.
\Question Exactly 1 solution if $k\neq 2$; infinite solutions if $k=2$; never no solution.
\Question Exactly 1 solution if $k\neq 2$; no solution if $k=2$; never infinite solutions.
\Question Exactly 1 solution for all $k$.
\end{Answer}

%-----------------------------------------------------
% index key words
%-----------------------------------------------------
\index{matrix!inverse}

%-----------------------------------------------------
% name, leave blank
% title, if the exercise has a name i.e. Hilbert's matrix
% difficulty = n, where n is the number of stars
% origin = "\cite{ref}"
%-----------------------------------------------------
\begin{Exercise}[
name={},
title={}, 
difficulty=0,
origin={\cite{KK}}]
Show $\left( AB\right) ^{-1}=B^{-1}A^{-1}$ by verifying that 
\begin{equation*}
AB\left(
B^{-1}A^{-1}\right) =I\text{ and }B^{-1}A^{-1}\left( AB\right) =I
\end{equation*}
\end{Exercise}

\begin{Answer}
$\left( AB\right)
B^{-1}A^{-1}=A\left( BB^{-1}\right) A^{-1}=AA^{-1}=I$ 
$B^{-1}A^{-1}\left(
AB\right) =B^{-1}\left( A^{-1}A\right) B=B^{-1}IB=B^{-1}B=I$
\end{Answer}

%-----------------------------------------------------
% index key words
%-----------------------------------------------------
\index{matrix inverse}
\index{matrix equation}

%-----------------------------------------------------
% name, leave blank
% title, if the exercise has a name i.e. Hilbert's matrix
% difficulty = n, where n is the number of stars
% origin = "\cite{ref}"
%-----------------------------------------------------
\begin{Exercise}[
name={},
title={}, 
difficulty=0,
origin={\cite{YL}}]
Solve for $X$ given that it satisfies
\[
DXD^T = \trace(BC)BC
\]
where
\[
B=
\begin{mat}
2 & 1 & 0\\
-3 & 4 & 0
\end{mat}\;
C=
\begin{mat}
2 & -1\\
3 & -2\\
1 & 0
\end{mat}
D=
\begin{mat}
2 & -2\\
1 & -2
\end{mat}.
\]

\end{Exercise}

\begin{Answer}
$
A=
\begin{mat}
0 & -1\\
-11 & -\frac{17}{2}
\end{mat}
$
\end{Answer}

%-----------------------------------------------------
% index key words
%-----------------------------------------------------
\index{matrix inverse}
\index{matrix equation}

%-----------------------------------------------------
% name, leave blank
% title, if the exercise has a name i.e. Hilbert's matrix
% difficulty = n, where n is the number of stars
% origin = "\cite{ref}"
%-----------------------------------------------------
\begin{Exercise}[
name={},
title={}, 
difficulty=0,
origin={\cite{YL}}]
Solve for $A$ given that it satisfies
\[
(I-A^T)^{-1} = (\trace(B)B^2)^T
\]
where
\[
B=
\begin{bmatrix}
2 & 3\\
1 & 2
\end{bmatrix}\;
\]

\end{Exercise}

\begin{Answer}
$
A=
\begin{mat}
-\frac34 & 3\\
1 & -\frac34
\end{mat}
$
\end{Answer}

%-----------------------------------------------------
% index key words
%-----------------------------------------------------
\index{matrix!inverse}
\index{matrix!trace of}

%-----------------------------------------------------
% name, leave blank
% title, if the exercise has a name i.e. Hilbert's matrix
% difficulty = n, where n is the number of stars
% origin = "\cite{ref}"
%-----------------------------------------------------
\begin{Exercise}[
name={},
title={}, 
difficulty=0,
origin={\cite{YL}}]
Find a formula for $\trace(A^{-1})$ where $A=\begin{bmatrix}a&b\\c&d\end{bmatrix}$ if $A$ is invertible.
\end{Exercise}

\begin{Answer}
$\trace{A^{-1}}=\frac{a+d}{ad-bc}$
\end{Answer}

%-----------------------------------------------------
% index key words
%-----------------------------------------------------
\index{cross product}
\index{orthogonal vector}

%-----------------------------------------------------
% name, leave blank
% title, if the exercise has a name i.e. Hilbert's matrix
% difficulty = n, where n is the number of stars
% origin = "\cite{ref}"
%-----------------------------------------------------
\begin{Exercise}[
name={},
title={}, 
difficulty=0,
origin={\cite{GHC}}]
Show, using the definition of the cross product, that $\vec u\cdot(\vec u\times\vec v)=0$; that is, that $\vec u$ is orthogonal to the cross product of $\vec u$ and $\vec v$.
\end{Exercise}

\begin{Answer}
With $\vec u = (u_1,u_2,u_3)$ and $\vec v = (v_1,v_2,v_3)$, we have
\begin{align*}
\vec u\cdot(\vec u\times\vec v) &= (u_1,u_2,u_3)\cdot (u_2v_3-u_3v_2,-(u_1v_3-u_3v_1),u_1v_2-u_2v_1) \\
		&= u_1(u_2v_3-u_3v_2) - u2(u_1v_3-u_3v_1)+u3(u_1v_2-u_2v_1)\\
		&=0.
\end{align*}

\end{Answer}

%-----------------------------------------------------
% index key words
%-----------------------------------------------------
\index{polynomial space}
\index{dimension}

%-----------------------------------------------------
% name, leave blank
% title, if the exercise has a name i.e. Hilbert's matrix
% difficulty = n, where n is the number of stars
% origin = "\cite{ref}"
%-----------------------------------------------------
\begin{Exercise}[
name={},
title={}, 
difficulty=0,
origin={\cite{JH}}]
Find a basis for, and the dimension of, \(  \polyspace_2 \).
\end{Exercise}

\begin{Answer}
One basis is \( \sequence{1,x,x^2} \), and so
the dimension is three.
\end{Answer}

%-----------------------------------------------------
% index key words
%-----------------------------------------------------
\index{determinant}
\index{Laplace expansion}
\index{cofactor expansion}

%-----------------------------------------------------
% name, leave blank
% title, if the exercise has a name i.e. Hilbert's matrix
% difficulty = n, where n is the number of stars
% origin = "\cite{ref}"
%-----------------------------------------------------
\begin{Exercise}[
name={},
title={}, 
difficulty=0,
origin={\cite{JH}}]
Evaluate the determinant
by performing a cofactor expansion 
    \begin{equation*}
      \begin{vmat}[r]
         3  &0  &1  \\
         1  &2  &2  \\
        -1  &3  &0
      \end{vmat}
    \end{equation*}
\Question along the first row,
\Question along the second row,
\Question along the third column.
\end{Exercise}
\begin{Answer}
\Question \( 3(+1)\begin{vmat}[r]
                          2  &2  \\
                          3  &0
                       \end{vmat}
                 +0(-1)\begin{vmat}[r]
                          1  &2  \\
                         -1  &0
                       \end{vmat}
                 +1(+1)\begin{vmat}[r]
                          1  &2  \\
                         -1  &3
                       \end{vmat} =-13 \)
\Question \( 1(-1)\begin{vmat}[r]
                          0  &1  \\
                          3  &0
                       \end{vmat}
                 +2(+1)\begin{vmat}[r]
                          3  &1  \\
                         -1  &0
                       \end{vmat}
                 +2(-1)\begin{vmat}[r]
                          3  &0  \\
                         -1  &3
                       \end{vmat} =-13 \)
\Question \( 1(+1)\begin{vmat}[r]
                          1  &2  \\
                         -1  &3
                       \end{vmat}
                 +2(-1)\begin{vmat}[r]
                          3  &0  \\
                         -1  &3
                       \end{vmat}
                 +0(+1)\begin{vmat}[r]
                          3  &0  \\
                          1  &2
                       \end{vmat} =-13 \)

\end{Answer}

%-----------------------------------------------------
% index key words
%-----------------------------------------------------
\index{Cramer's Rule}

%-----------------------------------------------------
% name, leave blank
% title, if the exercise has a name i.e. Hilbert's matrix
% difficulty = n, where n is the number of stars
% origin = "\cite{ref}"
%-----------------------------------------------------
\begin{Exercise}[
name={},
title={}, 
difficulty=0,
origin={\cite{SZ}}]
Use Cramer's Rule to solve for $x_{\mbox{\tiny$4$}}$.
\begin{multicols}{2}
\Question $\begin{array}{rcr} x_{\mbox{\tiny$1$}} - x_{\mbox{\tiny$3$}} & = & -2 \\ 
2x_{\mbox{\tiny$2$}} - x_{\mbox{\tiny$4$}} & = & 0  \\  
x_{\mbox{\tiny$1$}} -  2x_{\mbox{\tiny$2$}} + x_{\mbox{\tiny$3$}} & = & 0 \\
-x_{\mbox{\tiny$3$}} + x_{\mbox{\tiny$4$}} & = & 1  \end{array}$ 
\Question $\begin{array}{rcr} 4x_{\mbox{\tiny$1$}} + x_{\mbox{\tiny$2$}} & = & 4 \\ 
x_{\mbox{\tiny$2$}} - 3x_{\mbox{\tiny$3$}} & = & 1  \\  
10x_{\mbox{\tiny$1$}} +x_{\mbox{\tiny$3$}} + x_{\mbox{\tiny$4$}} & = & 0 \\
-x_{\mbox{\tiny$2$}} + x_{\mbox{\tiny$3$}} & = & -3  \end{array}$ 
\EndCurrentQuestion
\end{multicols}
\end{Exercise}

\begin{Answer}
\Question $4$
\Question $-1$
\end{Answer}

%-----------------------------------------------------
% index key words
%-----------------------------------------------------
\index{vector space}
\index{polynomial space}

%-----------------------------------------------------
% name, leave blank
% title, if the exercise has a name i.e. Hilbert's matrix
% difficulty = n, where n is the number of stars
% origin = "by name \cite{ref}"
%-----------------------------------------------------
\begin{Exercise}[
name={},
title={}, 
difficulty=0,
origin={\cite{JH}}]
For each, list three elements and then show it is a vector space.

\Question The set of linear polynomials
        \( \polyspace_1=\set{a_0+a_1x\suchthat a_0,a_1\in\Re} \) under the
        usual polynomial addition and scalar multiplication operations.

\Question The set of linear polynomials
        \( \set{a_0+a_1x\suchthat a_0-2a_1=0} \), under the
        usual polynomial addition and scalar multiplication operations.

\end{Exercise}

\begin{Answer}
\Question $1+2x$, $2-1x$, and $x$.
\Question $2+1x$, $6+3x$, and $-4-2x$.


\end{Answer}

%-----------------------------------------------------
% index key words
%-----------------------------------------------------
\index{augmented matrix}


%-----------------------------------------------------
% name, leave blank
% title, if the exercise has a name i.e. Hilbert's matrix
% difficulty = n, where n is the number of stars
% origin = "by name \cite{ref}"
%-----------------------------------------------------
\begin{Exercise}[
name={},
title={}, 
difficulty=0,
origin={\cite{GH}}]
Convert the given system of linear equations into an augmented matrix.
\Question
\[
\begin{linsys}{3}
3x&+&4y&+&5z&=&7\\
-x&+&y&-&3z&=&1\\
2x&-&2y&+&3z&=&5\\
\end{linsys}
\]

\Question
\[
\begin{linsys}{3}
2x&+&5y&-&6z&=&2\\
9x&&&-&8z&=&10\\
-2x&+&4y&+&z&=&-7\\
\end{linsys}
\]

\Question
\[
\begin{linsys}{4}
x_1& +&3x_2&-&4x_3& + &5x_4 &=&17 \\
-x_1&&&+&4x_3&+&8x_4 &=&1\\
2x_1&+&3x_2&+&4x_3&+&5x_4&=&6
\end{linsys}
\]

\Question
\[
\begin{linsys}{2}
3x_1 &-&2x_2&=&4 \\
2x_1 &&&=&3\\
-x_1&+&9x_2&=&8\\
5x_1&-&7x_2&=&13\\
\end{linsys}
\]

\end{Exercise}

\begin{Answer}
\Question 
$
\begin{amat}{3}
3&4&5&7\\-1&1&-3&1\\2&-2&3&5\\
\end{amat}
$
\Question 
$
\begin{amat}{3}
2&5&-6&2\\9&0&-8&10\\-2&4&1&-7\\
\end{amat}
$
\Question
$
\begin{amat}{4}
1&3&-4&5&17\\-1&0&4&8&1\\ 2&3&4&5&6
\end{amat}
$

\Question
$
\begin{amat}{2}
3&-2&4\\ 2&0&3\\-1&9&8\\5&-7&13
\end{amat}
$
\end{Answer}

%-----------------------------------------------------
% index key words
%-----------------------------------------------------
\index{reduced row echelon form}
\index{solution set}

%-----------------------------------------------------
% name, leave blank
% title, if the exercise has a name i.e. Hilbert's matrix
% difficulty = n, where n is the number of stars
% origin = "by name \cite{ref}"
%-----------------------------------------------------
\begin{Exercise}[
name={},
title={}, 
difficulty=0,
origin={\cite{SZ}}]
The following matrices are in reduced row echelon form.  Determine the solution of the corresponding system of linear equations or state that the system is inconsistent.  


\begin{multicols}{2}
\Question $
\begin{amat}{2}
1 & 0 & -2 \\ 
0 & 1 & 7  \\ 
\end{amat}
$
\Question $
\begin{amat}{3}
1 & 0 & 0 & -3 \\ 
0 & 1 & 0 & 20 \\ 
0 & 0 & 1 & 19  
\end{amat}
$
\Question $
\begin{amat}{4}
1 & 0 & 0 & 3 & 4 \\ 
0 & 1 & 0 & 6 & -6 \\ 
0 & 0 & 1 & 0 & 2 
\end{amat}
$
\Question $
\begin{amat}{4}
1 & 0 & 0 & 3 & 0 \\ 
0 & 1 & 2 & 6 & 0 \\ 
0 & 0 & 0 & 0 & 1 
\end{amat}
$
\Question $
\begin{amat}{4}
1 & \hphantom{-}0 & -8 & 1 & 7 \\ 
0 & 1 & 4 & -3 & 2 \\ 
0 & 0 & 0 & 0 & 0 \\
0 & 0 & 0 & 0 & 0 
\end{amat}
$
\Question $
\begin{amat}{3}
1 & \hphantom{-}0 & 9 & -3 \\ 
0 & 1 & -4 & 20 \\ 
0 & 0 & 0 & 0  
\end{amat}
$
\EndCurrentQuestion
\end{multicols}
\end{Exercise}

\begin{Answer}
\Question $(-2, 7)$
\Question $(-3, 20, 19)$
\Question $(-3t + 4, -6t - 6, 2, t)$ for all real numbers $t$
\Question Inconsistent
\Question $(8s - t + 7, -4s + 3t + 2, s, t)$ \\ for all real numbers $s$ and $t$
\Question $(-9t - 3, 4t + 20, t)$ \\ for all real numbers $t$
\end{Answer}

%-----------------------------------------------------
% index key words
%-----------------------------------------------------
\index{determinant}

%-----------------------------------------------------
% name, leave blank
% title, if the exercise has a name i.e. Hilbert's matrix
% difficulty = n, where n is the number of stars
% origin = "\cite{ref}"
%-----------------------------------------------------
\begin{Exercise}[
name={},
title={}, 
difficulty=0,
origin={\cite{KK}}]
Prove or disprove: If $A$ and $B$ are square matrices of the same size then $\detop \left( A+B\right) =\detop \left( A\right) +\detop
\left( B\right)$.
\end{Exercise}
\begin{Answer}
The statement is false.  Consider 
$A=
\begin{mat}
1 & 0 \\
0 & 1
\end{mat}
B=
\begin{mat}
-1 & 0 \\
0 & -1
\end{mat}.$
\end{Answer}

%-----------------------------------------------------
% index key words
%-----------------------------------------------------
\index{matrix inverse}
\index{matrix equation}


%-----------------------------------------------------
% name, leave blank
% title, if the exercise has a name i.e. Hilbert's matrix
% difficulty = n, where n is the number of stars
% origin = "\cite{ref}"
%-----------------------------------------------------
\begin{Exercise}[
name={},
title={}, 
difficulty=0,
origin={\cite{YL}}]
Given
\[
A=
\begin{mat}
2 & 2 & 0\\
4 & 3 & 0\\
3 & 2 & \frac12
\end{mat}.
\]
\Question Find $A^{-1}$.
\Question Solve for $X$ where $AX=B$ and 
\[
B=
\begin{mat}
1  & 0 & \frac12 & 0 & 0\\
0  & 1 & 0 & 2 & -1\\
-4 & 2 & \frac12 & 0 & 0
\end{mat}
\]
\Question Find $\left(\trans{\left(\frac12A\right)}\right)^{-1}$ if possible.
\end{Exercise}

\begin{Answer}
\Question
$
A=
\begin{mat}
-\frac32 & 1 & 0\\
2 & -1 & 0\\
1 & -2 & 2
\end{mat}
$
\Question
$
X=
\begin{mat}
-\frac32 & 1 & -\frac34 & 2 & -1\\
2 & -1 & 1 & -2 & 1\\
-7 & 2 & \frac32 & -4 & 2
\end{mat}
$
\Question
$
\begin{mat}
-3 & 4 & 2\\
2 & -2 & -4\\
0 & 0 & 4
\end{mat}
$
\end{Answer}

%-----------------------------------------------------
% index key words
%-----------------------------------------------------
\index{determinant}
\index{matrix!adjoint}

%-----------------------------------------------------
% name, leave blank
% title, if the exercise has a name i.e. Hilbert's matrix
% difficulty = n, where n is the number of stars
% origin = "\cite{ref}"
%-----------------------------------------------------
\begin{Exercise}[
name={},
title={}, 
difficulty=0,
origin={\cite{YL}}]
Prove: If $A$ is an invertible $\nbyn{n}$ matrix then 
$\det(\adj(A))=(\det(A))^{n-1}$.
\end{Exercise}
\begin{Answer}
Hint: Use the identity $\det(A)A^{-1}=\adj(A)$.
\end{Answer}

%-----------------------------------------------------
% index key words
%-----------------------------------------------------
\index{Cramer's Rule}

%-----------------------------------------------------
% name, leave blank
% title, if the exercise has a name i.e. Hilbert's matrix
% difficulty = n, where n is the number of stars
% origin = "\cite{ref}"
%-----------------------------------------------------
\begin{Exercise}[
name={},
title={}, 
difficulty=0,
origin={\cite{YL}}]
Solve only for $x_1$ using Cramer's Rule.
\[
\begin{linsys}{3}
x_1 & - & 2x_2 & +  & 3x_3 & = 4\\
   &   & 5x_2 & -  & 6x_3 & = 7\\
   &   &      &    & 8x_3 & = 9
\end{linsys}
\]
\end{Exercise}

\begin{Answer}
$x_1=4$

\end{Answer}

%-----------------------------------------------------
% index key words
%-----------------------------------------------------
\index{matrix!elementary}
\index{matrix!inverse}

%-----------------------------------------------------
% name, leave blank
% title, if the exercise has a name i.e. Hilbert's matrix
% difficulty = n, where n is the number of stars
% origin = "\cite{ref}"
%-----------------------------------------------------
\begin{Exercise}[
name={},
title={}, 
difficulty=0,
origin={\cite{KK}}]
Given matrices $A$ and $B$ and suppose a row operation is applied to $A$ and the result is $B$. Find the elementary matrix $E$ such that $EA = B$. Find the inverse of $E$, $E^{-1}$, such that $E^{-1}B = A$.
\Question $A=
\begin{mat}
1 & 2 & 1  \\
0 & 5 & 1 \\
2 & -1 & 4
\end{mat}\text{ and }
B=\begin{mat}
1 & 2 & 1\\
2 & -1 & 4 \\
0 & 5 & 1  
\end{mat}$
\Question $A=
\begin{mat}
1 & 2 & 1  \\
0 & 5 & 1 \\
2 & -1 & 4
\end{mat}\text{ and }
B=\begin{mat}
1 & 2 & 1\\
0 & 5 & 1 \\
1 & -\vspace{0.05in}\frac{1}{2} & 2  
\end{mat}$
\Question $A=
\begin{mat}
1 & 2 & 1  \\
0 & 5 & 1 \\
2 & -1 & 4
\end{mat}\text{ and }
B=\begin{mat}
1 & 2 & 1\\
2 & 4 & 5 \\
2 & -1 & 4  
\end{mat}$
\end{Exercise}
\begin{Answer}
\Question $E=E^{-1} =
\begin{mat}
1 & 0 & 0 \\
0 & 0 & 1 \\
0 & 1 & 0
\end{mat}$
\Question $E=
\begin{mat}
1 & 0 & 0 \\
0 & 1 & 0 \\
0 & 0 & \frac12
\end{mat},\;
E^{-1} =
\begin{mat}
1 & 0 & 0 \\
0 & 1 & 0 \\
0 & 0 & 2
\end{mat}$
\Question $E=
\begin{mat}
1 & 0 & 0 \\
0 & 1 & 1 \\
0 & 0 & 1
\end{mat},\;
E^{-1} =
\begin{mat}
1 & 0 & 0 \\
0 & 1 & -1 \\
0 & 0 & 1
\end{mat}$
\end{Answer}

%-----------------------------------------------------
% index key words
%-----------------------------------------------------
\index{elementary matrix}
\index{row-equivalent}

%-----------------------------------------------------
% name, leave blank
% title, if the exercise has a name i.e. Hilbert's matrix
% difficulty = n, where n is the number of stars
% origin = "\cite{ref}"
%-----------------------------------------------------
\begin{Exercise}[
name={},
title={}, 
difficulty=0,
origin={\cite{YL}}]
Show that
\[
A = 
\begin{mat}
5 & 7 & 9\\
1 & 2 & 3\\
4 & 5 & 6
\end{mat}
\]
and
\[
B =
\begin{mat}
1 & 2 & 3\\
4 & 5 & 6\\
8 & 10 & 12
\end{mat}
\]
are row-equivalent by finding 3 elementary matrices $E_i$ such that $E_3E_2E_1A=B$.
\end{Exercise}

\begin{Answer}
$
E_1=
\begin{mat}
0 & 1 & 0\\
1 & 0 & 0\\
0 & 0 & 1
\end{mat}
E_2=
\begin{mat}
1 & 0 & 0\\
0 & 1 & 0\\
0 & 0 & 2
\end{mat}
E_3=
\begin{mat}
1 & 0 & 0\\
-1 & 1 & 0\\
0 & 0 & 1
\end{mat}
$
Note: The answer is not unique.
\end{Answer}

%-----------------------------------------------------
% index key words
%-----------------------------------------------------
\index{matrix inverse}
\index{matrix equation}

%-----------------------------------------------------
% name, leave blank
% title, if the exercise has a name i.e. Hilbert's matrix
% difficulty = n, where n is the number of stars
% origin = "\cite{ref}"
%-----------------------------------------------------
\begin{Exercise}[
name={},
title={}, 
difficulty=0,
origin={\cite{YL}}]
Solve of $X$ given that it satisfies
\[
DXD^T = \trace(BC)BC
\]
where
\[
B=
\begin{mat}
2 & 1 & 0\\
-3 & 4 & 0
\end{mat}\;
C=
\begin{mat}
2 & -1\\
3 & -2\\
1 & 0
\end{mat}
D=
\begin{mat}
2 & -2\\
1 & -2
\end{mat}.
\]

\end{Exercise}

\begin{Answer}
$
A=
\begin{mat}
0 & -1\\
-11 & -\frac{17}{2}
\end{mat}
$
\end{Answer}

%-----------------------------------------------------
% index key words
%-----------------------------------------------------
\index{adjoint matrix}
\index{matrix!adjoint}
\index{matrix!invertible}
\index{invertible matrix}

%-----------------------------------------------------
% name, leave blank
% title, if the exercise has a name i.e. Hilbert's matrix
% difficulty = n, where n is the number of stars
% origin = "\cite{ref}"
%-----------------------------------------------------
\begin{Exercise}[
name={},
title={}, 
difficulty=0,
origin={\cite{YL}}]
Consider the matrix:
\[
A=\begin{mat}
\sin\theta&0&\cos\theta\\
0&\pi&0\\
-\cos\theta&0&\sin\theta
\end{mat}
\]
\Question Determine the value(s) of $\theta$ for which $A$ is invertible.
\Question Determine the adjoint of $A$.
\end{Exercise}

\begin{Answer}
\Question Determinant of $A$ is $\pi\neq0$, therefore $A$ is invertible.
\Question $\begin{mat}
\pi\sin\theta & 0 & -\pi\cos\theta\\
0 & 1 & 0\\
\pi\cos\theta & 0 & \pi\sin\theta
\end{mat}$
\end{Answer}

%-----------------------------------------------------
% index key words
%-----------------------------------------------------
\index{graphical interpretation}


%-----------------------------------------------------
% name, leave blank
% title, if the exercise has a name i.e. Hilbert's matrix
% difficulty = n, where n is the number of stars
% origin = "by name \cite{ref}"
%-----------------------------------------------------
\begin{Exercise}[
name={},
title={}, 
difficulty=0,
origin={\cite{KK}}]
Graphically, find the point of intersection of the two lines 
$3x+y=3$ and $x+2y=1.$ That is, graph each line
and see where they intersect. 
\end{Exercise}

\begin{Answer}
$\begin{linsys}{3}
3x&+&y&=&3 \\
x&+&2y&=&1
\end{linsys}
$, Solution is: $x=1,i\;y=0$
\end{Answer}

%-----------------------------------------------------
% index key words
%-----------------------------------------------------
\index{matrix!inverse}

%-----------------------------------------------------
% name, leave blank
% title, if the exercise has a name i.e. Hilbert's matrix
% difficulty = n, where n is the number of stars
% origin = "\cite{ref}"
%-----------------------------------------------------
\begin{Exercise}[
name={},
title={}, 
difficulty=0,
origin={\cite{YL}}]
Show that if $A$ is invertible then its inverse is unique.
\end{Exercise}

\begin{Answer}
Suppose that that there exists two inverse then show that they are equal using the fact that they are both the inverse of $A$.
\end{Answer}

%-----------------------------------------------------
% index key words
%-----------------------------------------------------
\index{solution set}

%-----------------------------------------------------
% name, leave blank
% title, if the exercise has a name i.e. Hilbert's matrix
% difficulty = n, where n is the number of stars
% origin = "by name \cite{ref}"
%-----------------------------------------------------
\begin{Exercise}[
name={},
title={}, 
difficulty=0,
origin={\cite{KK}}]
If a system of linear equations has fewer equations than variables and
there exist a solution to this system. Is it possible that
your solution is the only one?\ Explain.
\end{Exercise}
\begin{Answer}
No.  There must be a free variable and since the system is consistent there are infinitely many solutions 
\end{Answer}

%-----------------------------------------------------
% index key words
%-----------------------------------------------------
\index{matrix!inverse}
\index{matrix!transpose}

%-----------------------------------------------------
% name, leave blank
% title, if the exercise has a name i.e. Hilbert's matrix
% difficulty = n, where n is the number of stars
% origin = "\cite{ref}"
%-----------------------------------------------------
\begin{Exercise}[
name={},
title={}, 
difficulty=0,
origin={\cite{KK}}]
Show that if $A$ is an invertible $n\times n$ matrix, then so is 
$\trans{A}$ and $\left(\trans{A}\right) ^{-1}=\trans{\left( A^{-1}\right)}.$ 
\end{Exercise}

\begin{Answer}
\begin{eqnarray*}
\trans{A}\trans{\left( A^{-1}\right)} &=&\trans{\left( A^{-1}A\right)}=\trans{I}=I \\
\trans{\left( A^{-1}\right)}\trans{A} &=&\trans{\left( AA^{-1}\right)}=\trans{I}=I
\end{eqnarray*}

\end{Answer}

%-----------------------------------------------------
% index key words
%-----------------------------------------------------
\index{matrix!elementary}
\index{matrix!transpose}
\index{matrix equation}

%-----------------------------------------------------
% name, leave blank
% title, if the exercise has a name i.e. Hilbert's matrix
% difficulty = n, where n is the number of stars
% origin = "\cite{ref}"
%-----------------------------------------------------
\begin{Exercise}[
name={},
title={}, 
difficulty=0,
origin={\cite{YL}}]
Show that there exist an $\nbyn{n}$ symmetric elementary matrix $E$ such that $E^2=I$.
\end{Exercise}

\begin{Answer}
Show that the elementary matrix obtained by interchanging the first and last row satisfy the conditions.
\end{Answer}

%-----------------------------------------------------
% index key words
%-----------------------------------------------------
\index{matrix!inverse}

%-----------------------------------------------------
% name, leave blank
% title, if the exercise has a name i.e. Hilbert's matrix
% difficulty = n, where n is the number of stars
% origin = "\cite{ref}"
%-----------------------------------------------------
\begin{Exercise}[
name={},
title={}, 
difficulty=0,
origin={\cite{YL}}]
Show that a matrix $A$ which satisfy $A^3+3A^2+A+I=0$ is invertible and express its inverse in terms of $A$ and the identity.
\end{Exercise}

\begin{Answer}
Show that $A^{-1}=-A^2-3A -I$.
\end{Answer}

%-----------------------------------------------------
% index key words
%-----------------------------------------------------
\index{matrix inverse}

%-----------------------------------------------------
% name, leave blank
% title, if the exercise has a name i.e. Hilbert's matrix
% difficulty = n, where n is the number of stars
% origin = "\cite{ref}"
%-----------------------------------------------------
\begin{Exercise}[
name={},
title={}, 
difficulty=0,
origin={\cite{YL}}]
Prove: If $AB$ and $BA$ are both invertible then $A$ and $B$ are both invertible.
\end{Exercise}

\begin{Answer}
Hint: Use the definition of the inverse of a matrix.
\end{Answer}

%-----------------------------------------------------
% index key words
%-----------------------------------------------------
\index{determinant}
\index{Laplace expansion}
\index{cofactor expansion}

%-----------------------------------------------------
% name, leave blank
% title, if the exercise has a name i.e. Hilbert's matrix
% difficulty = n, where n is the number of stars
% origin = "\cite{ref}"
%-----------------------------------------------------
\begin{Exercise}[
name={},
title={}, 
difficulty=0,
origin={\cite{JH}}]
    Which real numbers \( \theta \) make
    \begin{equation*}
       \begin{mat}
          \cos\theta  &-\sin\theta  \\
          \sin\theta  &\cos\theta
       \end{mat}
    \end{equation*}
    equal to zero?

\end{Exercise}
\begin{Answer}
There are no real numbers \( \theta \) that make the matrix singular 
      because the determinant of the matrix
      \( \cos^2\theta+\sin^2\theta \) is never $0$, it equals $1$
      for all $\theta$.

\end{Answer}

%-----------------------------------------------------
% index key words
%-----------------------------------------------------
\index{matrix inverse}

%-----------------------------------------------------
% name, leave blank
% title, if the exercise has a name i.e. Hilbert's matrix
% difficulty = n, where n is the number of stars
% origin = "\cite{ref}"
%-----------------------------------------------------
\begin{Exercise}[
name={},
title={}, 
difficulty=0,
origin={\cite{YL}}]
Prove: If $A$ and $B$ are square matrices satisfying $AB=I$, then $A=B^{-1}$.
\end{Exercise}

\begin{Answer}
Hint: Show that the homogeneous system $Ax=0$ has only the trivial solution.
\end{Answer}


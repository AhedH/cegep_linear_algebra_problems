%-----------------------------------------------------
% index key words
%-----------------------------------------------------
\index{matrix!inverse}
\index{matrix!nilpotent}

%-----------------------------------------------------
% name, leave blank
% title, if the exercise has a name i.e. Hilbert's matrix
% difficulty = n, where n is the number of stars
% origin = "\cite{ref}"
%-----------------------------------------------------
\begin{Exercise}[
name={},
title={}, 
difficulty=0,
origin={\cite{JH}}]
    Show that if \( T \) is square and if \( T^4 \) is the zero matrix
    then \( (I-T)^{-1}=I+T+T^2+T^3 \).
\end{Exercise}
\begin{Answer}
      Checking that when $I-T$ is multiplied on both sides by that expression
      (assuming that $T^4$ is the zero matrix) then the result is the 
      identity matrix is easy.
      The obvious generalization is that if \( T^n \) is the zero matrix
      then \( (I-T)^{-1}=I+T+T^2+\cdots+T^{n-1} \); the check again is
      easy.  
\end{Answer}

%-----------------------------------------------------
% index key words
%-----------------------------------------------------
\index{trace of a matrix}

%-----------------------------------------------------
% name, leave blank
% title, if the exercise has a name i.e. Hilbert's matrix
% difficulty = n, where n is the number of stars
% origin = "\cite{ref}"
%-----------------------------------------------------
\begin{Exercise}[
name={},
title={}, 
difficulty=0,
origin={\cite{YL}}]
Prove the following statements
\Question If $A$ is a $\nbyn{n}$ matrix then $\trace(\trans{A})=\trace(A)$.
\Question If $A$ is a $\nbyn{n}$ matrix then $\trace(cA)=c\trace(A)$.
\Question If $A$ and $B$ are $\nbyn{n}$ matrices then $\trace(A+B)=\trace(A)+\trace(B)$.
\Question If $A$ and $B$ are $\nbyn{n}$ matrices then $\trace(AB)=\trace(BA)$.
\end{Exercise}

\begin{Answer}
\Question Hint: The main diagonal is remains the same if a matrix is transposed.
\Question Hint: Apply the definition of the trace to arbitrary matrices $cA$ and $A$. 
\Question Hint: Apply the definition of the trace to arbitrary matrices $A$ and $B$.
\Question Hint: Analyse the ij product of the elements of the main diagonal.
\end{Answer}

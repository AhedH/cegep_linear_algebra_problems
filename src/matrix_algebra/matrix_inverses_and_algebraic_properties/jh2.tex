%-----------------------------------------------------
% index key words
%-----------------------------------------------------
\index{matrix!transpose}

%-----------------------------------------------------
% name, leave blank
% title, if the exercise has a name i.e. Hilbert's matrix
% difficulty = n, where n is the number of stars
% origin = "\cite{ref}"
%-----------------------------------------------------
\begin{Exercise}[
name={},
title={}, 
difficulty=0,
origin={\cite{JH}}]
\Question Show that \( \trans{(G+H)}=\trans{G}+\trans{H} \).
\Question Show that \( \trans{(r\cdot H)}=r\cdot\trans{H} \).
\Question Show that \( \trans{(GH)}=\trans{H}\trans{G} \).
\Question Show that the matrices \( H\trans{H} \) and \( \trans{H}H \) are symmetric.
\end{Exercise}

\begin{Answer}
        \Question The \( i,j \) entry of \( \trans{(G+H)} \) is
          \( g_{j,i}+h_{j,i} \).
          That is also the \( i,j \) entry of \( \trans{G}+\trans{H} \).
        \Question The \( i,j \) entry of \( \trans{(r\cdot H)} \) is
          \( rh_{j,i} \),
          which is also the \( i,j \) entry of \( r\cdot\trans{H} \).
        \Question  The \( i,j \) entry of \( \trans{GH} \) is the $j,i$ entry
          of $GH$, which is the dot product of the
          \( j \)-th row of \( G \) and the \( i \)-th column of \( H \).
          The \( i,j \) entry of \( \trans{H}\trans{G} \) is the dot product of
          the \( i \)-th row of \( \trans{H} \) and the \( j \)-th column of
          \( \trans{G} \), which is the
          dot product of the \( i \)-th column of \( H \) and the
          \( j \)-th row of \( G \).
          Dot product is commutative and so these two are equal.
        \Question By the prior part each equals its transpose, e.g.,
          $\trans{(H\trans{H})}=\trans{\trans{H}}\trans{H}=H\trans{H}$.
\end{Answer}

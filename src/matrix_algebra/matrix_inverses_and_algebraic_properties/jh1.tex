%-----------------------------------------------------
% index key words
%-----------------------------------------------------
\index{matrix!algebraic properties}

%-----------------------------------------------------
% name, leave blank
% title, if the exercise has a name i.e. Hilbert's matrix
% difficulty = n, where n is the number of stars
% origin = "\cite{ref}"
%-----------------------------------------------------
\begin{Exercise}[
name={},
title={}, 
difficulty=0,
origin={\cite{JH}}]
Prove each, assuming that the operations are defined, 
where \( G \), \( H \), and 
\( J \) are matrices, where
\( Z \) is the zero matrix, and where \( r \) and \( s \) are scalars.
\Question Matrix addition is commutative \( G+H=H+G \).
\Question Matrix addition is associative \( G+(H+J)=(G+H)+J \).
\Question The zero matrix is an additive identity \( G+Z=G \).
\Question \( 0\cdot G=Z \)
\Question \( (r+s)G=rG+sG \)
\Question Matrices have an additive inverse \( G+(-1)\cdot G=Z \).
\Question \( r(G+H)=rG+rH \)
\Question \( (rs)G=r(sG) \)
\end{Exercise}

\begin{Answer}
     First, each of these properties
      is easy to check in an entry-by-entry way.
      For example, writing
      \begin{equation*}
        G=
        \begin{mat}
          g_{1,1}  &\ldots  &g_{1,n}  \\
          \vdots   &        &\vdots   \\
          g_{m,1}  &\ldots  &g_{m,n}       
        \end{mat}
        \qquad
        H=
        \begin{mat}
          h_{1,1}  &\ldots  &h_{1,n}  \\
          \vdots   &        &\vdots   \\
          h_{m,1}  &\ldots  &h_{m,n}       
        \end{mat}
      \end{equation*}
      then, by definition we have
      \begin{equation*}
        G+H=
        \begin{mat}
          g_{1,1}+h_{1,1}  &\ldots  &g_{1,n}+h_{1,n}  \\
          \vdots           &        &\vdots           \\
          g_{m,1}+h_{m,1}  &\ldots  &g_{m,n}+h_{m,n}       
        \end{mat}
      \end{equation*}
      and
      \begin{equation*}
        H+G=
        \begin{mat}
          h_{1,1}+g_{1,1}  &\ldots  &h_{1,n}+g_{1,n}  \\
          \vdots           &        &\vdots           \\
          h_{m,1}+g_{m,1}  &\ldots  &h_{m,n}+g_{m,n}       
        \end{mat}
      \end{equation*}
      and the two are equal since their entries are equal 
      $g_{i,j}+h_{i,j}=h_{i,j}+g_{i,j}$.

\end{Answer}

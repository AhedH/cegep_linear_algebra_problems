%-----------------------------------------------------
% index key words
%-----------------------------------------------------
\index{determinant}

%-----------------------------------------------------
% name, leave blank
% title, if the exercise has a name i.e. Hilbert's matrix
% difficulty = n, where n is the number of stars
% origin = "\cite{ref}"
%-----------------------------------------------------
\begin{Exercise}[
name={},
title={}, 
difficulty=0,
origin={\cite{JH}}]
Prove or disprove:     The determinant is a linear function, that is
    \( \det(x\cdot T+y\cdot S)=x\cdot \det(T)+y\cdot \det(S) \).
\end{Exercise}

\begin{Answer}
Disprove.       Recall that constants come out one row at a time.
      \begin{equation*}
         \det(
         \begin{mat}[r]
            2  &4  \\
            2  &6  \\
         \end{mat})
         =
         2\cdot\det(\begin{mat}[r]
            1  &2  \\
            2  &6  \\
         \end{mat})
         =
         2\cdot 2\cdot \det(\begin{mat}[r]
            1  &2  \\
            1  &3  \\
         \end{mat})
      \end{equation*}
      This contradicts linearity (here we didn't need \( S \), i.e., we can 
      take $S$ to be the matrix of zeros). 
\end{Answer}

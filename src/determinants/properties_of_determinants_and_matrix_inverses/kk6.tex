%-----------------------------------------------------
% index key words
%-----------------------------------------------------
\index{determinant}
\index{matrix!of cofators}
\index{matrix!adjoint}
\index{matrix!inverse}

%-----------------------------------------------------
% name, leave blank
% title, if the exercise has a name i.e. Hilbert's matrix
% difficulty = n, where n is the number of stars
% origin = "\cite{ref}"
%-----------------------------------------------------
\begin{Exercise}[
name={},
title={}, 
difficulty=0,
origin={\cite{KK}}]
Determine whether the matrix $A$ has an inverse by finding whether the
determinant is non zero. If the determinant is nonzero, find the inverse
using the formula for the inverse which involves the cofactor matrix.
\begin{multicols}{2}
\Question $\begin{mat}
1 & 2 & 3 \\
0 & 2 & 1 \\
3 & 1 & 0
\end{mat}$
\Question $\begin{mat}
1 & 2 & 0 \\
0 & 2 & 1 \\
3 & 1 & 1
\end{mat}$
\Question $\begin{mat}
1 & 3 & 3 \\
2 & 4 & 1 \\
0 & 1 & 1
\end{mat}$
\Question $\begin{mat}
1 & 0 & 3 \\
1 & 0 & 1 \\
3 & 1 & 0
\end{mat}$
\EndCurrentQuestion
\end{multicols}
\end{Exercise}

\begin{Answer}
\Question $
\begin{detmat}
1 & 2 & 3 \\
0 & 2 & 1 \\
3 & 1 & 0
\end{detmat} = -13$ and so it has an inverse given by
\begin{eqnarray*}
\frac{1}{-13}\trans{
\begin{mat}
\begin{detmat}
2 & 1 \\
1 & 0
\end{detmat}
& - 
\begin{detmat}
0 & 1 \\
3 & 0
\end{detmat}
&
\begin{detmat}
0 & 2 \\
3 & 1
\end{detmat}\\
-
\begin{detmat}
2 & 3 \\
1 & 0
\end{detmat}
& 
\begin{detmat}
1 & 3 \\
3 & 0
\end{detmat}
& -
\begin{detmat}
1 & 2 \\
3 & 1
\end{detmat}\\
\begin{detmat}
2 & 3 \\
2 & 1
\end{detmat}
& -
\begin{detmat}
1 & 3 \\
0 & 1
\end{detmat}
& 
\begin{detmat}
1 & 2 \\
0 & 2
\end{detmat}
\end{mat}}\\ =\frac{1}{-13}
\trans{
\begin{mat}
-1 & 3 & -6 \\
3 & -9 & 5 \\
-4 & -1 & 2
\end{mat}} \\
=
\begin{mat}
\vspace{0.05in}\frac{1}{13} & -\vspace{0.05in}\frac{3}{13} & \vspace{0.05in}\frac{4}{13} \\
-\vspace{0.05in}\frac{3}{13} & \vspace{0.05in}\frac{9}{13} & \vspace{0.05in}\frac{1}{13} \\
\vspace{0.05in}\frac{6}{13} & -\vspace{0.05in}\frac{5}{13} & -\vspace{0.05in}\frac{2}{13}
\end{mat}
\end{eqnarray*}

\Question $
\begin{detmat}
1 & 2 & 0 \\
0 & 2 & 1 \\
3 & 1 & 1
\end{detmat} = 7$ so it has an inverse. This inverse is $\frac{1}{7}
\trans{
\begin{mat}
1 & 3 & -6 \\
-2 & 1 & 5 \\
2 & -1 & 2
\end{mat}} = 
\begin{mat}
\vspace{0.05in}\frac{1}{7} & -\vspace{0.05in}\frac{2}{7} & \vspace{0.05in}\frac{2}{7} \\
\vspace{0.05in}\frac{3}{7} & \vspace{0.05in}\frac{1}{7} & -\vspace{0.05in}\frac{1}{7} \\
-\vspace{0.05in}\frac{6}{7} & \vspace{0.05in}\frac{5}{7} & \vspace{0.05in}\frac{2}{7}
\end{mat}$

\Question $
\begin{detmat}
1 & 3 & 3 \\
2 & 4 & 1 \\
0 & 1 & 1
\end{detmat} = 3$
so it has an inverse which is
$
\begin{mat}
1 & 0 & -3 \\
-\vspace{0.05in}\frac{2}{3} & \vspace{0.05in}\frac{1}{3} & \vspace{0.05in}\frac{5}{3} \\
\vspace{0.05in}\frac{2}{3} & -\vspace{0.05in}\frac{1}{3} & -\vspace{0.05in}\frac{2}{3}
\end{mat}
$

\Question $
\begin{detmat}
1 & 0 & 3 \\
1 & 0 & 1 \\
3 & 1 & 0
\end{detmat}= 2$
and so it has an inverse. The inverse is 
$
\begin{mat}
-\vspace{0.05in}\frac{1}{2} & \vspace{0.05in}\frac{3}{2} & 0 \\
\vspace{0.05in}\frac{3}{2} & -\vspace{0.05in}\frac{9}{2} & 1 \\
\vspace{0.05in}\frac{1}{2} & -\vspace{0.05in}\frac{1}{2} & 0
\end{mat}$

\end{Answer}

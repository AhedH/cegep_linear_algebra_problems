%-----------------------------------------------------
% index key words
%-----------------------------------------------------
\index{determinant}
\index{matrix!inverse}

%-----------------------------------------------------
% name, leave blank
% title, if the exercise has a name i.e. Hilbert's matrix
% difficulty = n, where n is the number of stars
% origin = "\cite{ref}"
%-----------------------------------------------------
\begin{Exercise}[
name={},
title={}, 
difficulty=0,
origin={\cite{KK}}]
Suppose $A,B$ are $\nbyn{n}$ matrices and that $AB=I$ then show that
$BA=I.$ \textit{Hint:} First explain why
$\det \left( A\right) ,\det \left( B\right) $ are both nonzero. Then $\left(
AB\right) A=A$ and then show $BA\left( BA-I\right) =0.$ From this use what
is given to conclude $A\left( BA-I\right) =0.$ 
\end{Exercise}

\begin{Answer}
$1=\det \left( A\right) \det \left( B\right) $.
Hence both $A$ and $B$ have inverses. Given any $X$,
\[
A\left( BA-I\right) X=\left( AB\right) AX-AX=AX-AX = 0
\]
and so it follows that $\left( BA-I\right)X=0.$ Since $X$ is arbitrary, it follows that $BA=I.$
\end{Answer}

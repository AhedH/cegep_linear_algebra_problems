%-----------------------------------------------------
% index key words
%-----------------------------------------------------
\index{determinant}
\index{adjoint matrix}
\index{matrix!adjoint}

%-----------------------------------------------------
% name, leave blank
% title, if the exercise has a name i.e. Hilbert's matrix
% difficulty = n, where n is the number of stars
% origin = "\cite{ref}"
%-----------------------------------------------------
\begin{Exercise}[
name={},
title={}, 
difficulty=0,
origin={\cite{JH}}]
Derive a formula for the adjoint of a diagonal matrix.

\end{Exercise}

\begin{Answer}
      Consider this diagonal matrix.
      \begin{equation*}
        D=
        \begin{mat}
          d_1  &0   &0   &\ldots    \\
          0    &d_2 &0   &          \\
          0    &0   &d_3            \\
               &    &    &\ddots    \\
               &    &    &      &d_n   
        \end{mat}
      \end{equation*}
      If $i\neq j$ then the $i,j$~minor is an $\nbyn{(n-1)}$ matrix
      with only $n-2$ nonzero entries, because we have deleted
      both $d_i$ and $d_j$.
      Thus, at least one row or column of the minor is all zeroes, and
      so the cofactor $D_{i,j}$ is zero.
      If $i=j$ then the minor is the diagonal matrix with entries
      $d_1$, \ldots, $d_{i-1}$, $d_{i+1}$, \ldots, $d_n$.
      Its determinant is obviously $(-1)^{i+j}=(-1)^{2i}=1$ 
      times the product of those.
      \begin{equation*}
        \adj(D)
        =
        \begin{mat}
          d_2\cdots d_n    &0                 &      &0    \\
          0                &d_1d_3\cdots d_n  &      &0    \\
                           &                  &\ddots      \\
                           &                  &       &d_1\cdots d_{n-1} 
        \end{mat}
      \end{equation*}

\end{Answer}
